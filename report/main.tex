\documentclass{article}

% Language setting
% Replace `english' with e.g. `spanish' to change the document language
\usepackage[english]{babel}

% Set page size and margins
% Replace `letterpaper' with `a4paper' for UK/EU standard size
\usepackage[letterpaper,top=2cm,bottom=2cm,left=3cm,right=3cm,marginparwidth=1.75cm]{geometry}

% Useful packages
\usepackage{amsmath}
\usepackage{graphicx}
\usepackage[colorlinks=true, allcolors=blue]{hyperref}

\title{Operations Reserach 2 Final Thesis}
\author{Youssef Ben Khalifa}

\begin{document}
\maketitle

\section{Introduction}

In this thesis we will go over the implementation of the main heuristics involving the famous TSP problem. The entire implementation is done using the C programming language and the CPLEX optimization LP optimization framework. 
\subsection{The Traveling Salesman Problem}
The Traveling Salesman Problem is among the most disucssed and researched problems in the field of Operations Research.

\paragraph{Mathematical Formulation}
The TSP can be formulated as an optimization problem as follows:
\begin{equation}
    \min \sum_{i=1}^{n} \sum_{j=1}^{n} c_{ij} x_{ij}
\end{equation}
subject to:
\begin{equation}
    \sum_{i=1}^{n} x_{ij} = 1 \quad \forall j \in \{1, \ldots, n\}
\end{equation}
\begin{equation}
    \sum_{j=1}^{n} x_{ij} = 1 \quad \forall i \in \{1, \ldots, n\}
\end{equation}
\begin{equation}
    u_i - u_j + nx_{ij} \leq n-1 \quad \forall i \in \{2, \ldots, n\}, j \in \{2, \ldots, n\}
\end{equation}
\begin{equation}
    x_{ij} \in \{0, 1\} \quad \forall i, j \in \{1, \ldots, n\}
\end{equation}
\begin{equation}
    u_i \in \mathit{text} \quad \forall i \in \{1, \ldots, n\}
\end{equation}


\section{TSP data structure}
Throught the implementation of the Metaheuristics and Mathheuristics, we will be using the following data structures:
\begin{itemize}
    \item \text{instance} : A structure that holds the TSP instance data. This is implemented as a C data structure 
    in which we hold all the metadata relevant to the loaded TSP instance.
    \item \text{solution} : A structure that holds the TSP solution data. Normally this is represented as a simple integer array in which we map for every nodex 
    index its successor in the tour. This type of solution representation suits both the indirected and directed graph cases.
\end{itemize}
The code for the data structures implementation can be found in the appendinx section.
\section{TSP Metaheuristics}

\subsection{Greedy Randomized Adaptive Search (GRASP)}

\subsection{Extra Mileage}

\subsection{Two-Opt}

\subsection{Tabu Search}


\section{TSP with CPLEX}

\subsection{Bender's subtour elimination method}

\subsection{Patching Heuristic}

\subsection{CPLEX Callback implementation}


\newpage

\bibliographystyle{alpha}
\bibliography{sample}


\end{document}